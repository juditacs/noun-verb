\documentclass[xcolor=rgb]{beamer}

\usepackage[utf8]{inputenc}

\usepackage[english,hungarian]{babel}
\usepackage{graphicx}
\title{MorphoDeep: Morfológiai elemzés deep learninggel}
\author[\'Acs Judit]{\'Acs Judit\\ \texttt{judit@aut.bme.hu}}
  \institute{BME AUT}
  \date{2016.~október 27.}

\begin{document}
  
\begin{frame}
\titlepage 
\end{frame}

\begin{frame}{Motiváció}
    \begin{itemize}
        \item Morfológia: szavak belső szerkezetének vizsgálata.
            \begin{itemize}
                \item \emph{házak = ház + (a)k}
                \item \emph{zúzalékával?}
            \end{itemize}
        \item A magyar morfológia kiemelkedően gazdag.
        \item Szinte minden nyelvtechnológiai feladathoz kell morfológiai elemzés.
        \item Jelenleg kézzel írt szabályok alapján történik.
        \item Deep learninget magyarra még nem alkalmaztak.
    \end{itemize}
\end{frame}

\begin{frame}{Irodalomkutatás}
    \begin{enumerate}
        \item szegmentálás
            \begin{itemize}
                \item HMM-alapú, MDL kritérium
                \item Morfessor (Creutz and Lagus, 2002)\nocite{Creutz:2002}
            \end{itemize}
        \item karakteralapú deep learning
            \begin{itemize}
                \item gépi fordítás (Lee et al., 2016)\nocite{Lee:2016}
            \end{itemize}
        \item morfológiai reinflexió
            \begin{itemize}
                \item SIGMORPHON Shared Task (Cotterell et al., 2016)\nocite{Cotterell:2016}
                \item 12 csapat beadása
            \end{itemize}
    \end{enumerate}
\end{frame}

\begin{frame}{Baseline: szófajok osztályzása szóalak alapján}
    \begin{itemize}
        \item szóalak vs.~ngram alapú
        \item túltanulás! - gyakorlatilag egy memóriát építek
        \item hogyan lehet általánosítani?
    \end{itemize}
\end{frame}

\begin{frame}{Megvalósítás}
    \begin{itemize}
        \item Architektúra
            \begin{itemize}
                \item jelenleg FFNN
                \item RNN, LSTM, CNN?
            \end{itemize}
        \item Technológia
        \begin{itemize}
            \item Python, jupyter
            \item Tensorflow, Keras
        \end{itemize}
    \end{itemize}
\end{frame}

\begin{frame}{Nehézségek}
    \begin{itemize}
        \item nagyon ritka az adat,
        \item morfémahatárok nem egyértelműek,
        \item kevés gold adat, sok silver,
        \pause
        \item magyar morfológia
    \end{itemize}
\end{frame}

\begin{frame}{Hivatkoz\'asok}
\bibliographystyle{plain}
\bibliography{ml}
\end{frame}

\end{document}
